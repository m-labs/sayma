\section{Clock distribution}

\subsection{Crate clock distribution}\label{crate-clock-distribution}

The crate distributes a 100MHz clock on a RTM RF backplane. This clock
is typically externally supplied from a high quality source, but it is
desirable to include a 100 MHz oscillator on the MCH RTM and on the
Sayma RTM the for turnkey/standalone operation (with limited timing
performance).

In a multi-crate system, all crates need to receive the same 100MHz
clock to support sample-accurate operation.

\subsection{RTIO}\label{rtio}

Sayma and Metlino shall include a general purpose XO of e.g.~125MHz,
connected to a general purpose FPGA clock input pin. This is a simple
addition that make the boards a bit friendlier to developers. It also
allows for debugging and bootstrapping of the clocks during development:
This XO becomes necessary if we use a transceiver PLL chip that needs to
be configured before it outputs a clock.

\subsubsection{Metlino}\label{metlino}

In root mode, the Metlino receives the 100MHz clock and turns it into a
200MHz RTIO clock that it uses as reference clock for its DRTIO
transmitters.

In satellite mode, the Metlino recovers the RTIO clock from the fiber.

The following clock resources should be available on Metlino to support
this operation:

\begin{itemize}

	\item
	Si5324 for 100-\textgreater{}200MHz in root mode, and CDR jitter
	filtering in satellite mode.
	\item
	Si5324 free-running based on local XO for providing a CDR reference in
	satellite mode.
\end{itemize}

The Si5324 shall have its two clock outputs connected to a transceiver
clock input (so that we can transmit back synchronously and at fixed
latency) and to a general purpose clock input on the FPGA.
Transceiver-fabric clock routing inside the FPGA is of poor quality, so
we want to mitigate that.

The Metlino will be double-width and connected to its RTM to receive the
100MHz RTM clock (required in root mode).

\subsubsection{Sayma}\label{sayma}

Sayma cards recover their RTIO clock from the backplane's transceiver
link or - if they are stand-alone -- from their SFP/SATA DRTIO
transciever link. This requires the same hardware as the Metlino in root
mode: Si5324 connected in the same way.

\subsection{DRTIO}\label{drtio}

DRTIO (distributed real-time input/output) achieves three distinct
things over a single high speed serial link:

\begin{itemize}

	\item
	It transfers the RTIO clock
	\item
	It transfers the RTIO time. This means that it will designate a
	specific RTIO clock cycle as timestamp zero.
	\item
	It transfers data. Data consists of RTIO events (outputs or inputs)
	and low bandwidth non-realtime auxiliary traffic.
\end{itemize}

%Note that the RTIO time (clock plus the cycle counter) is the primary
%and authoritative source of time in the ARTIQ tree. The RTIO clock is
%however not an extremely low noise clock that could serve as the sample
%clock in data conversion or as a base clock for picosecond level
%timestamping. Having another ``better'' clock do these tasks is not
%trivial since the alignment between such a sample clock and the RTIO
%clock is unknown. When data is transferred between the two clock domains
%it is undefined which RTIO cycle corresponds to which sample clock
%cycle.

\subsection{JESD204 synchronization
	procedure}\label{jesd204-synchronization-procedure}

While JESD204B subclass 1 provides ``fixed latency'' for the data
transfer between a converter (ADC or DAC) and the FPGA, this is
fundamentally insufficient for DRTIO. We need more than just fixed
latency. A JESD204B link has two deviceclocks: one for the converter and
one for the FPGA. The SYSREF signal is used to designate which cycle of
the faster of the two deviceclocks corresponds to the beginning of a
cycle in the slower deviceclock. The slower deviceclock and SYSREF have
an a priori unknown phase with respect to the RTIO clock.

Timestamping a certain sample to a specific RTIO cycle requires two
things in addition to JESD204B subclass 1 deterministic latency:

\begin{itemize}

	\item
	Reproducible alignment of the sample clock with the RTIO clock. This
	is guaranteed by fixed latencies in the DRTIO branch of the clocking
	(master oscillator -\textgreater{} MCH RTM -\textgreater{} Metlino
	-\textgreater{} AMC backplane DRTIO link -\textgreater{} Sayma AMC)
	and in the sample branch (master oscillator -\textgreater{} MCH RTM
	-\textgreater{} RF backplane -\textgreater{} Sayma RTM -\textgreater{}
	PLL -\textgreater{} clock distribution -\textgreater{} DAC/ADC). This
	also requires the backplane clock and the sample clock to be integer
	multiples of the RTIO clock.
	\item
	Reproducible alignment of SYSREF and the slower FPGA deviceclock to
	the RTIO clock. This is done actively.
\end{itemize}

The FPGA shall align SYSREF with designated RTIO clock edges. The
alignment should be better than a DAC clock cycle and reproducible
across reboots.

The FPGA first roughly aligns SYSREF within one cycle before a desired
RTIO clock edge by asserting the synchronization signal of the clock
chip, which resets its dividers. This alignment is optional and may have
an uncertainty of several DAC clock cycles. It is only used to decrease
the required scan range of the delay elements used in the next steps.

The FPGA then analyzes SYSREF by repeatedly sampling it with the RTIO
clock while scanning a calibrated I/O input delay. This measures the
SYSREF phase with a high precision.

The delay scan mechanism is limited by the resolution and stability of
the scan element. The resolution must be significantly smaller than a
DAC clock period. There are three delay elements available to perform
the scan:

\begin{itemize}

	\item
	IDELAYE3 in the FPGA. Uncertainty about PVT effects.
	\item
	Digital delay in the clock distribution chip. Infinite delay, low
	noise.
	\item
	Analog delay in the clock distribution chip (HMC704X only, not
	AD9516-1). Very fine and well calibrated, but too noisy to be used on
	a sample clock.
\end{itemize}

We plan to use the latter two elements for the scan.

The FPGA then rounds the phase to an integer multiple of sample clock
cycles using previously stored fractional delay data (delay \textless{}-
round(measured - fractional)) and stores the new fractional delay
(fractional \textless{}- measured - delay). It now programs the digital
phase shifters of the slower clocks (FPGA deviceclock and SYSREF) with
the negative of the rounded delay value.

This technique can be implemented on the AD9154 FMC cards, using the
digital delay of the AD9516-1 and IDELAYE3.

\subsection{Sayma RTM clock chip
	connections}\label{sayma-rtm-clock-chip-connections}

The HMC7044 has 14 outputs. They are used for:

\begin{itemize}

	\item
	DAC1 deviceclock
	\item
	DAC1 SYSREF
	\item
	DAC2 deviceclock
	\item
	DAC2 SYSREF
	\item
	ADC1 deviceclock
	\item
	ADC1 SYSREF
	\item
	ADC2 deviceclock
	\item
	ADC2 SYSREF
	\item
	FPGA SYSREF {[}with fine delay{]}
	\item
	FPGA MGT reference clock for DAC
	\item
	FPGA MGT reference clock for ADC
	\item
	additional outputs to FPGA, usable e.g.~if we have problems with the
	recovered RTIO clock.
\end{itemize}

\subsection{Clock constraints}\label{clock-constraints}

\subsubsection{Constraints}\label{constraints}

\begin{itemize}

	\item
	t\_RTIO = n * 1ns
	
	\begin{itemize}

		\item
		period of the coarse RTIO clock
		\item
		n integer to avoid rounding errors and beating between RTIO clock
		and user habit
		\item
		n not necessarily a power of two
		\item
		the same throughout the ARTIQ tree to avoid beating of channels
	\end{itemize}
	\item
	t\_DRTIO\_link = n * 10 * t\_RTIO with n being 1, 2, 4, 8
	
	\begin{itemize}

		\item
		line period of the DRTIO link
		\item
		due to 8b10b and parallel bus width
		\item
		n not a power of two could work but looks impractical.
		\item
		does not need to be the same n for each link
		\item
		AMC backplane links can probably not to 10 GHz line rate but 5 GHz,
		fibers (SFP+) can
	\end{itemize}
	\item
	t\_SAWG\_DATA = t\_RTIO/
\end{itemize}

\begin{verbatim}
f_DAC/f_SAWG: {1, 2, 4, 8}
f_SAWG/f_RTIO: {1, 2, 4, 8}
f_RTIO/f_DRTIO: {10, 20, 40}
f_JESD_P/f_RTIO: {1, 2}
f_JESD/f_JESD_P: {40}
\end{verbatim}

\begin{longtable}[]{@{}lllllll@{}}

	(GHz) & f\_DAC & f\_SAWG & f\_JESD\_P & f\_JESD & f\_RTIO &
	f\_DRTIO\tabularnewline

	\endhead
	A & 2.4 & 0.6 & 0.15 & 6 & 0.15 & 3\tabularnewline
	B & 2 & 1 & 0.25 & 10 & 0.125 & 5\tabularnewline
	C & 0.3 & 0.3 & 0.15 & 6 & 0.15 & 3\tabularnewline

\end{longtable}
